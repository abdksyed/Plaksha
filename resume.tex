\documentclass{resume} 
\usepackage[left=0.75in,top=0.6in,right=0.75in,bottom=0.6in]{geometry} 
\usepackage{hyperref}
\newcommand{\tab}[1]{\hspace{.2667\textwidth}\rlap{#1}}
\newcommand{\itab}[1]{\hspace{0em}\rlap{#1}}
\name{Syed Abdul Khader} % Your name 
\address{(+91)7013542783 \\ syed.abdul@plaksha.edu.in}


\begin{document}

\begin{rSection}{Education}


{\bf Plaksha University} \hfill {\em August 2022 -- Present} 
\\ Post Graduate Diploma - AI/ML\hfill { Current CGPA: 9.7 }

{\bf JNTUH College of Engineering} \hfill {\em September 2014 -- May 2019} 
\\ Bachelors in Technology + M.B.A.\hfill { Overall Percentage: 81\% }

\end{rSection}

\begin{rSection}{Work Experience}
	\begin{rSubsection}{Infinstor Pvt. Ltd., India}{June 2021 - July 2022}{Data Scientist I}{}
	 \item Developed Object Detection and Classification models for a Client to extract text from documents.
	 \item Performed Language Modelling to convert the handwritten forms in digital format, and reduce the turn around time of the procces by 90\%.
	 \item Developed LogBERT model for SQL queries anomaly detection and published an explainer on \href{https://medium.com/infinstor/logbert-log-file-anomaly-detection-using-bert-an-explainer-db20bfd2f91f}{medium}.
	 \item Architected the pipeline for User Churn Prediction on a large tabular dataset.
	\end{rSubsection}
	\begin{rSubsection}{Infor, India}{June 2019 - April 2021}{Dev. Business Analyst}{}
	 \item Automating Customer Data Migration Tasks, reducing turn over time from 2 days to minutes.
	 \item Integration of various Infor ERP systems to MSCRM/Salesforce.
	\end{rSubsection}
	
\end{rSection}

\begin{rSection}{Projects}
{\bf Segmentation of Pulmonary Artery During Robotic Right Lower Lobectomy}
\\This project aims developing a computer vision-based approach to reliably identify the pulmonary artery during robotic right lower lobectomy. The base model used is a Mask-RCNN to segment the arteries in the video frames. The work is published in $103^{rd}$ annual meeting of American Association for Thoracic Surgery(AATS).

{\bf Panoptic Segmenation of Constuction Objects using Transformer Model}
\\As part of a community, we annotated 50 classes of data, like a road grader, aac block, concrete pump etc. We than trained a transformer model, particularly DETR to initially perform object detection and then extend it to perform panoptic segmentation. The detailed explanation and results are available at \href{https://github.com/abdksyed/DETR}{github}.


\end{rSection}

\begin{rSection}{Technical Strengths}

\begin{tabular}{ @{} >{\bfseries}l @{\hspace{6ex}} l }
Languages \ & Python, Java, SQL  \\
Technologies & Deep Learning, Machine Learning, MLOps\\
Libraries and Frameworks & PyTorch, Numpy, sklearn, pandas\\
Version Control & Github
\end{tabular}

\end{rSection}

\end{document}